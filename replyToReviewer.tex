\documentclass[12pt,a4paper]{article}
\usepackage[utf8]{inputenc}
\usepackage{amsmath}
\usepackage{amsfonts}
\usepackage{amssymb}
\begin{document}

Please find our reply to the points raised by reviewer 1.

\begin{quotation}
General comments:
In this well-written and timely Applications Note, Hadfield and colleagues present the Nextstrain platform – a web-based visualization and analysis platform for viral genome phylogenies. This reviewer has used the site on many occasions to explore viral genomic data, and can attest to the fact that Nextstrain is an incredibly useful resource to the viral genomics and public health communities. Its thoughtful, intuitive interface makes the tool easy to use, even for individuals with no formal bioinformatics training. Nextstrain has played a role in a number of recent outbreaks of public health importance, from Ebola to Zika, and is an excellent resource for annual surveillance of influenza viruses. The platform's commitment to open science is exceptional, and demonstrates to the viral genomics community how powerful data-sharing can be.

\paragraph{}
This is an excellent Applications Note describing a fantastic tool, and I have no hesitations in recommending the manuscript for publication. I have no major revisions to suggest, and only very minor style suggestions that the authors may take or leave:

\end{quotation}

We thank the reviewer for these kind words!


\begin{quote}
1. Page 1, Line 39: "ongoing and ever-present" is somewhat redundant; I'd just go with "ever-present"

2. Page 1, Line 40: perhaps change "recent epidemics" to "recent events" so "epidemics" isn't repeated 3x in the sentence.
\end{quote}

We have changed the wording of the manuscript to address these points. 

\begin{quote}
3. Page 1, Line 42: it might be helpful to point the reader to a general reference on phylodynamics, assuming the reference limit hasn't already been reached.
\end{quote}


4. Page 1, Line 43 (column 2): could references to the three databases (papers or URLs) be included here?

5. Page 2, Line 35: if it can be explained in a very short sentence, it might be interesting to mention the caveat of "nearly all outbreaks" - which ones aren't amenable to Nextstrain-ification?

6. Page 2, Line 51: it should be noted that the transmission inference depends on how well the outbreak in question was sampled - if data is missing, important transmission links could be missed.

7. Page 2, Line 20 (column 20): is there an example of one of these mutations that could be mentioned here?

There is a fair amount of passive voice in the manuscript - popping the text into readable.io to find these instances and changing the construction to make them active will make the writing a little more lively (and it usually frees up some words in the word count).

\end{document}