\documentclass[12pt,a4paper]{article}
\usepackage[utf8]{inputenc}
\usepackage{amsmath}
\usepackage{amsfonts}
\usepackage{amssymb}
\begin{document}

We thank the reviewer for their kind words, and have addressed all of the points (detailed below).

\begin{quote}
1. Page 1, Line 39: ``ongoing and ever-present" is somewhat redundant; I'd just go with ``ever-present"

2. Page 1, Line 40: perhaps change ``recent epidemics" to ``recent events" so ``epidemics" isn't repeated 3x in the sentence.
\end{quote}
We have changed the wording of the manuscript to address these points.

\begin{quote}
3. Page 1, Line 42: it might be helpful to point the reader to a general reference on phylodynamics, assuming the reference limit hasn't already been reached.
\end{quote}
We have reworded this section to include a reference to ``Viral Phylodynamics'', Voltz \textit{et al.} 2013.

\begin{quote}
4. Page 1, Line 43 (column 2): could references to the three databases (papers or URLs) be included here?
\end{quote}
We have added URLs to the text.

\begin{quote}
5. Page 2, Line 35: if it can be explained in a very short sentence, it might be interesting to mention the caveat of ``nearly all outbreaks" - which ones aren't amenable to Nextstrain-ification?
\end{quote}
There are indeed pathogens for which the general approach outlined in this paper is insufficient, and so we have changed the text to read ``Nextstrain is able to be extended to most outbreaks with readily accessible genomic data, although we note the potential for recombination or low mutation rate to confound phylogenetic signal.''


\begin{quote}
6. Page 2, Line 51: it should be noted that the transmission inference depends on how well the outbreak in question was sampled - if data is missing, important transmission links could be missed.
\end{quote}
This is a very important point, and we have added the following sentence to the manuscript: ``It is important to note that sampling bias can obscure transmission links, and in certain cases we have chosen not to display the inferred states due to this bias.''

\begin{quote}
7. Page 2, Line 20 (column 20): is there an example of one of these mutations that could be mentioned here?
\end{quote}
We have added a reference to the 2017 publication ``Three mutations switch H7N9 influenza to human-type receptor specificity'' (de Vries \textit{et al.}), however have not listed these mutations as we detect none in the (consensus) sequences shown on nextstrain.org

\begin{quote}
There is a fair amount of passive voice in the manuscript - popping the text into readable.io to find these instances and changing the construction to make them active will make the writing a little more lively (and it usually frees up some words in the word count).
\end{quote}
Thankyou for alerting me us to this website. We have removed 11/12 usages of the passive voice, as well as simplifying sentences with too many syllables where possible.

\paragraph{}
Sincerely,

James Hadfield


\end{document}