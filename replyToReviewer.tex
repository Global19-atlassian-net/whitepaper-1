\documentclass[11pt,oneside,letterpaper]{article}

% graphicx package, useful for including eps and pdf graphics
\usepackage{graphicx}
\DeclareGraphicsExtensions{.pdf,.png,.jpg}

% basic packages
\usepackage{color}
\usepackage{parskip}
\usepackage{float}
\usepackage{hyperref}

% text layout
\usepackage{geometry}
\geometry{textwidth=16.25cm} % 15.25cm for single-space, 16.25cm for double-space
\geometry{textheight=22cm} % 22cm for single-space, 22.5cm for double-space

% helps to keep figures from being orphaned on a page by themselves
\renewcommand{\topfraction}{0.85}
\renewcommand{\textfraction}{0.1}

% bold the 'Figure #' in the caption and separate it with a period
% Captions will be left justified
\usepackage[labelfont=bf,labelsep=period,font=small]{caption}

% review layout with double-spacing
%\usepackage{setspace}
%\doublespacing
%\captionsetup{labelfont=bf,labelsep=period,font=doublespacing}

% cite package, to clean up citations in the main text. Do not remove.
\usepackage{cite}
%\renewcommand\citeleft{(}
%\renewcommand\citeright{)}
%\renewcommand\citeform[1]{\textsl{#1}}

% Remove brackets from numbering in list of References
\renewcommand\refname{\large References}
\makeatletter
\renewcommand{\@biblabel}[1]{\quad#1.}
\makeatother

\usepackage{authblk}
\renewcommand\Authands{ \& }
\renewcommand\Authfont{\normalsize \bf}
\renewcommand\Affilfont{\small \normalfont}
\makeatletter
\renewcommand\AB@affilsepx{, \protect\Affilfont}
\makeatother

% notation
\usepackage{amsmath}
\usepackage{amssymb}

\begin{document}

\newgeometry{top=4cm}

Dear Bioinformatics editorial board,

Thank you for conducting a review of our manuscript entitled ``Nextstrain: real-time tracking of pathogen evolution''.  We have responded to the reviewer comments and believe the manuscript is now suitable for publication in Bioinformatics.

Point-by-point responses follow, as well as a PDF showing revisions that have been made since the last submission.

Sincerely,\\
James Hadfield

\restoregeometry

\newpage

\section*{Reviewer responses}

Original reviewer criticisms are in plain text.  Our responses follow in \textbf{bold}.

We thank the reviewer for their kind words, and have addressed all of the points (detailed below).

1. Page 1, Line 39: ``ongoing and ever-present'' is somewhat redundant; I'd just go with ``ever-present''

2. Page 1, Line 40: perhaps change ``recent epidemics'' to ``recent events'' so ``epidemics'' isn't repeated 3x in the sentence.

\textbf{We have changed the wording of the manuscript to address these points.}

3. Page 1, Line 42: it might be helpful to point the reader to a general reference on phylodynamics, assuming the reference limit hasn't already been reached.

\textbf{We have reworded this section to include a reference to ``Viral Phylodynamics'', Voltz \textit{et al.} 2013.}

4. Page 1, Line 43 (column 2): could references to the three databases (papers or URLs) be included here?

\textbf{We have added URLs to the text.}

5. Page 2, Line 35: if it can be explained in a very short sentence, it might be interesting to mention the caveat of ``nearly all outbreaks'' -- which ones aren't amenable to Nextstrain-ification?

\textbf{There are indeed pathogens for which the general approach outlined in this paper is insufficient, and so we have changed the text to read ``Nextstrain is able to be extended to most outbreaks with readily accessible genomic data, although we note the potential for recombination or low mutation rate to confound phylogenetic signal.''}

6. Page 2, Line 51: it should be noted that the transmission inference depends on how well the outbreak in question was sampled - if data is missing, important transmission links could be missed.

\textbf{This is a very important point, and we have added the following sentence to the manuscript: ``It is important to note that sampling bias can obscure transmission links, and in certain cases we have chosen not to display the inferred states due to this bias.''}

7. Page 2, Line 20 (column 20): is there an example of one of these mutations that could be mentioned here?

\textbf{We have added a reference to the 2017 publication ``Three mutations switch H7N9 influenza to human-type receptor specificity'' (de Vries \textit{et al.}).}

There is a fair amount of passive voice in the manuscript -- popping the text into readable.io to find these instances and changing the construction to make them active will make the writing a little more lively (and it usually frees up some words in the word count).

\textbf{Thank you for alerting us to this website. We have removed 11/12 usages of the passive voice, as well as simplifying sentences with too many syllables where possible.}

\end{document}
